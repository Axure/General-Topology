\documentclass[11pt]{article}
\usepackage{amsmath}
\usepackage{amsfonts}
\usepackage{amssymb}
\usepackage{amsthm}
\usepackage{graphicx}

\newcommand{\dd}{\mathrm{d}}
\newcommand{\pd}{\partial}
\newcommand{\mr}{\mathbb{R}}

\newtheorem{problem}{Problem}
\numberwithin{problem}{section}
\title{Topological Spaces and Continuous Functions}
\author{Hu Zheng \\ Department of Mathematics, Zhejiang University}
\newenvironment{solution}
               {\let\oldqedsymbol=\qedsymbol
                \renewcommand{\qedsymbol}{$\blacktriangleleft$}
                \begin{proof}[\bfseries\upshape Solution:]}
               {\end{proof}
                \renewcommand{\qedsymbol}{\oldqedsymbol}}

\begin{document}

\maketitle

\section{Basis for a Topology}

\section{The Subspace Topology}

\begin{problem}
A map $f:X\rightarrow Y$ is said to be an \textbf{\textit{open map}} if for every open set $U$ of $X$, the set $f(U)$ is open in $Y$. Show that $\pi_1: X\times Y\rightarrow X$ and $\pi_2: X\times Y\rightarrow X$ are open maps.



\begin{solution}



\end{solution}
\end{problem}

\begin{problem}
Show that the dictionary order topology on the set $\mr\times\mr$ is the same as the product topology $\mr_d\times\mr$, where $\mr_d$ denotes $\mr$ in the discrete topology. Compare this topology with the standard topology on $\mr^2$.
\begin{proof}
We would like to show that there exist a homeomorphism between these two spaces. That their open subsets correspond or contain each other.

Let's recall the definition of dictionary order topology. It is a kind of order topology. There is no largest or smallest element in $\mr\times \mr$ so the only elements in the basis is of the type form $(a, b)$. Write $a=(x_a, y_a), b=(x_b, y_b)$, we have $x_b \ge x_a$ an d if $x_b = x_a$ then $y_b > y_a$.

And recall the definition of the product topology. For a finite index set, the basis of the product consists of products of basis, which are further reduced to $\{x\}\times U$ where $U$ is in the basis of $\mr$, which should be an open interval $(y,y')$. Such kinds of open set is contained in the basis of the dictionary order topology, as it's just $((x, y), (x, y'))$. With this kind of open sets we are able to constitute $x\times(-\infty, y), x\times(y, +\infty), x\times(-\infty, +\infty)$ and then easily other kind of open sets in the basis of the dictionary order topology.

Are there any geometric intuitions behind the scenes?

\end{proof}
\end{problem}


\end{document}
